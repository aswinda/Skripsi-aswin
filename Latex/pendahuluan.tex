%----------------------------------------------------------------------------------------
%	PENDAHULUAN
%----------------------------------------------------------------------------------------
\section*{PENDAHULUAN} % Sub Judul PENDAHULUAN
% Tuliskan isi Pendahuluan di bagian bawah ini. 
% Jika ingin menambahkan Sub-Sub Judul lainnya, silakan melihat contoh yang ada.
% Sub-sub Judul 
\subsection*{Latar Belakang}
Pertumbuhan pengguna internet memberikan dampak langsung terhadap penggunaan media sosial. Asosiasi Penyelenggara Jasa Internet Indonesia (APJII) mengungkapkan jumlah pengguna internet di Indonesia mencapai 88.1 juta orang hingga akhir tahun 2014. Data survei ini menyatakan bahwa ada tiga alasan utama orang Indonesia menggunakan internet. Tiga alasan itu adalah untuk mengakses sarana sosial/komunikasi (72$\%$), sumber informasi harian (65$\%$), dan mengikuti perkembangan jaman (51$\%$). Tiga alasan utama mengakses internet itu dipraktikan melalui empat kegiatan utama, yaitu menggunakan jejaring sosial (87$\%$), mencari informasi (69$\%$), instant messaging (60$\%$) dan mencari berita terbaru (60$\%$).

Twitter merupakan salah satu jejaring sosial yang sering digunakan sebagai alat komunikasi , Twitter juga dimanfaatkan sebagai media untuk promosi, dan kampanye politik. Twitter merupakan jaringan sosial berupa \textit{microlog} serta memiliki karakteristik dan format yang unik dengan simbol ataupun aturan khusus. Disebut \textit{microblog} karena penggunanya hanya dapat mengirim dan membaca pesan blog seperti pada umumnya dengan batas maksimal sejumlah 140 karakter, pesan tersebut dikenal dengan \textit{tweet} (\citeauthor{ZHANG2011} \cite*{ZHANG2011}). Setiap \textit{tweet} yang di posting pengguna beraneka ragam sesuai dengan keinginan pengguna. Postingan itu bisa berupa pendapat, saran, ataupun kritikan tentang topik-topik tertentu. Keanekagaraman postingan tersebut serta banyaknya penggunaan bahasa yang tidak baku pada \textit{tweet} menjadi alasan diperlukan analisis sentimen.

Analisis sentimen dapat disebut juga \textit{opinion mining} yang bertujuan untuk menganalisis, memahami, mengolah, dan mengestrak data tekstual yang berupa opini terhadap entitas seperti organisasi dan topik tertentu agar mendapatkan suatu informasi (\citeauthor{LIU2010} \cite*{LIU2010}). Analisis sentimen juga berfokus pada pengolahan opini yang mengandung polaritas, yaitu memiliki nilai sentimen yang positif ataupun negatif (Novantirani, 2014).Masalah yang ada dalam analisis sentimen biasanya sulit dalam mendefinisikan, menentukan konsep masalah, sub masalah, dan tujuan yang berfungsi sebagai kerangka kinerja dalam berbagai penelitian (\citeauthor{LIU2010} \cite*{LIU2010}).

Metode klasifikasi sentimen dapat digunakan untuk menentukan sentimen dari sebuah tweet. Metode klasifikasi Naïve Bayes classifier dapat dibagi menjadi dua, yaitu \textit{multi-variate Bernoulli} dan \textit{Multinomial Naïve Bayes} (\citeauthor{MANNING2008} \cite*{MANNING2008}). Dalam penelitian ini metode klasifikasi yang digunakan adalah \textit{Multinomial  Naïve Bayes}. \textit{Multinomial Naïve Bayes} digunakan karena proses yang sederhana dan mudah diaplikasikan pada berbagai keadaan sehingga tidak akan mengalami kegagalan secara keseluruhan pada hasilnya (\citeauthor{MANNING2008}, \cite*{MANNING2008}). Membuat model klasifikasi diperlukan sebuah training set atau data latih yang akan menghasilkan \textit{term} atau fitur sebagai penciri. Namun tidak semua \textit{term} dapat dijadikan sebagai penciri, sehingga perlu dilakukan seleksi fitur. Tujuan utama dari seleksi fitur adalah efisiensi \textit{term} yang dihasilkan sebagai penciri serta peningkatan akurasi untuk klasifikasi (\citeauthor{MANNING2008} \cite*{MANNING2008}). Seleksi fitur yang digunakan dalam penelitian ini adalah \textit{Mutual Information }(MI). 

Menurut \citeauthor{DIMASTYO2014} (\cite*{DIMASTYO2014}) dalam penelitiannya dalam klasifikasi dokumen \textit{email} dengan menggunakan metode peluang\textit{ Multinomial Naïve Bayes} maka seleksi fitur dengan \textit{supervised feature selection} yaitu MI lebih bagus dalam melakukan klasifikasi dokumen \textit{spam} dibandingkan dengan \textit{unsupervised feature selection} yaitu \textit{Inverse Document Frequency} (IDF). \citeauthor{ADITYAWAN2014} (\cite*{ADITYAWAN2014}) telah melakukan analisis sentimen dengan proses klasifikasi pada \textit{tweet} menjadi 3 sentimen yaitu positif, negatif, dan netral menggunakan data seimbang. Data yang digunakan terdiri atas 8 entitas berbeda dengan masing-masing entitas setiap sentimennya terdiri atas 80-90 data. Hasil akurasi dari klasifikasi data tweet dengan menggunakan metode \textit{Multinomial  Naïve Bayes} adalah 66.42$\%$. Adapun nilai akurasi tiap sentimennya untuk model Multinomial yaitu 58.62$\%$ untuk sentimen positif, 77.42$\%$ untuk sentimen negatif, 64.40$\%$ untuk sentimen netral.

Penelitian ini menggunakan data \textit{tweet} bahasa Indonesia yang akan diklasifikasikan kedalam tiga kelas sentimen positif, negative, dan netral menggunakan seleksi fitur \textit{Mutual Information} dengan menggunakan metode \textit{Multinomial Naïve Bayes}.

% Sub-sub Judul 
\subsection*{Perumusan Masalah}
Rumusan malasah dalam penelitian ini adalah:
\begin{enumerate}[noitemsep] 
	\item Bagaimana mengklasifikasikan sentimen  pada data Twitter menggunakan seleksi fitur \textit{Mutual Information}(MI) dan \textit{Inverse Document Frequency} dengan metode \textit{Multinomial Naïve Bayes}?
	\item Apakah seleksi fitur dengan menggunakan MI mampu meningkatkan akurasi jika dibandingkan dengan IDF?
\end{enumerate}

\subsection*{Tujuan}
Tujuan dari penelitian ini adalah:
\begin{enumerate}[noitemsep] 
	\item Mengklasifikasikan sentimen  pada data tweet menggunakan seleksi fitur MI dan IDF dengan metode \textit{Multinomial Naïve Bayes}.
	\item Dapat membandingkan hasil akurasi menggunakan seleksi fitur MI dengan IDF.
\end{enumerate}

\subsection*{Manfaat}
Manfaat dari penelitian ini adalah:
\begin{enumerate}[noitemsep] 
	\item Mengetahui cara mengklasifikasikan sentimen pada data tweet menggunakan seleksi fitur MI dan IDF dengan metode \textit{Multinomial Naïve Bayes} dalam melakukan analisis sentimen.
	\item Mengetahui hasil akurasi klasifikasi menggunakan seleksi fitur MI dan IDF dengan metode \textit{Multinomial Naïve Bayes} dalam melakukan analisis sentimen.
\end{enumerate}

\subsection*{Ruang Lingkup}
Ruang lingkup penelitian adalah:
\begin{enumerate}[noitemsep] 
	\item Metode klasifikasi yang digunakan adalah \textit{Multinomial Naive Bayes}.
	\item Seleksi fitur yang digunakan adalah MI dan IDF.  
	\item Penelitian ini menggunakan data Twitter dengan kata kunci "kementerian" dan "pendidikan".
\end{enumerate}

