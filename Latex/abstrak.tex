%----------------------------------------------------------------------------------------
%	ABSTRACT
%----------------------------------------------------------------------------------------
\Abstract{\scriptsize 
% ---- Tuliskan abstrak di bagian ini seperti contoh.
Analisis Sentimen dalam penelitian ini akan menggunakan data twitter berbahasa Indonesia. Data yang digunakan adalah data yang berkaitan dengan kementrian pendidikan dengaa kata kunci "kementrian", "pendidikan", "sekolah", "indonesia". Data yang digunakan dalam penelitian berjumlah 6000, dan untuk data latihnya akan dibersi sentimen secara manual. Klasifikasi yang akan digunakan dalam penelitian adalah \textit{multinomial naive bayes}. \textit{Negation Handling} memiliki peran untuk menambah akurasi dari klasifikasi sentimen. Dalam seleksi fitur pada penelitian ini akan ditambahkan metode \textit{Negation Handling} dan diimplementasikan dengan data Twitter berbahasa Indonesia. Pada penelitian ini akan membandingkan apakah dengan penambahan metode \textit{Negation Handling} pada data Twitter bahasa Indonesia dapat meningkatkan akurasi daripada tanpa menggunakan metode \textit{Negation Handling}.
% ---- Akhir bagian abstrak
\normalsize}
