%----------------------------------------------------------------------------------------
%	ABSTRACT
%----------------------------------------------------------------------------------------
\Abstract{\scriptsize 
% ---- Tuliskan abstrak di bagian ini seperti contoh.
Penelitian ini bertujuan untuk mengklasifikasi data \textit{tweet} pada Twiiter menjadi 3 sentimen yaitu positif, negatif, dan netral. Data yang digunakan terdiri atas 6000 data yang diperoleh dari tags.hawksey.info. Sebelum tahap klasifikasi, dilakukan beberapa tahap yaitu \textit{indexing} seperti tokenisasi, normalisasi kata, pembuangan \textit{stopwords} dan \textit{stemming} serta pemilihan fitur menggunakan \textit{Inverse Document Frequency} dan \textit{Mutual Information}. Data yang dihasilkan setelah proses \textit{indexing} dibagi  menjadi  dua \textit{subset} data yang terdiri dari 70 persen data latih dan 30 persen data uji. Data latih akan digunakan dalam tahapan pemilihan fitur sementara data uji digunakan untuk melakukan pengujian terhadap sistem klasifikasi yang telah dibuat dengan mengggunakan metode \textit{Multinomial Naïve Bayes}. Adapun manfaat dari penlitian ini adalah mengetahui hasil akurasi terbaik antara metode \textit{Multinomial Naïve Bayes} menggunakan seleksi fitur \textit{Mutual Information} dengan \textit{Inverse Document Frequency}  dalam melakukan analisis sentimen.
% ---- Akhir bagian abstrak
\normalsize}
